% Created 2017-05-03 Wed 02:14
% Intended LaTeX compiler: pdflatex
\documentclass[11pt]{article}
\usepackage[utf8]{inputenc}
\usepackage[T1]{fontenc}
\usepackage{graphicx}
\usepackage{grffile}
\usepackage{longtable}
\usepackage{wrapfig}
\usepackage{rotating}
\usepackage[normalem]{ulem}
\usepackage{amsmath}
\usepackage{textcomp}
\usepackage{amssymb}
\usepackage{capt-of}
\usepackage{hyperref}
\usepackage{minted}
\author{Jeremy}
\date{\today}
\title{}
\hypersetup{
 pdfauthor={Jeremy},
 pdftitle={},
 pdfkeywords={},
 pdfsubject={},
 pdfcreator={Emacs 25.1.1 (Org mode 9.0.6)}, 
 pdflang={English}}
\begin{document}

\tableofcontents

\section{Electron}
\label{sec:orgb9e4dda}

How to run and package Threepenny apps with \href{https://electron.atom.io}{Electron} and \href{https://github.com/electron-userland/electron-packager\#electron-packager}{Electron Packager}.

For reference, a minimal working example is available \href{https://github.com/barischj/threepenny-gui-electron-example}{here}.

\subsection{Justification}
\label{sec:org7377b81}

Normally when running a Threepenny app we execute our Haskell, with \texttt{stack exec}
or otherwise, which starts a local server and we open our browser on a certain
port to view our app. However this has a few subtle drawbacks.

One drawback is that most browsers are designed with remote servers in mind and
have security features which don't make much sense for local server
architectures like Threepenny's. Take file selection for example. When a user
selects a file in a browser, the browser only exposes the file name and contents
to the server, not allowing the server to receive the full file path. However
for local server architectures there isn't much point in sending the entire file
contents to the server since the server is on the same file system and could
read the file directly, if only it had the full file path. Electron displays our
app in a Chromium instance with many of these security features removed.

Another drawback is that the user has to run the app from the command line.
Using electron-packager we can package native apps for Linux, macOS and Windows.

\subsection{Running with Electron}
\label{sec:org562eb1f}
To run a Threepenny app with Electron we need an Electron \href{https://electron.atom.io/docs/tutorial/quick-start/\#main-process}{main process}. We
provide this one: \href{./electron/electron.js}{electron.js}. It runs the following on startup: - selects a
free port to run on - executes our Threepenny app binary, passing the port to
run on as an argument - waits for Threepenny's server to start accepting
connections
\begin{itemize}
\item opens an Electron window which loads the URL of our Threepenny app
\end{itemize}

To get started with the linked \texttt{electron.js} first add this \href{'./electron/package.json'}{package.json} to your
project root directory. You'll need Node installed and \texttt{npm} on your PATH,
confirm by running \texttt{which npm}. Now run \texttt{npm install} from the project root
directory to install the necessary dependencies.

The linked \texttt{electron.js} executes the Threepenny app binary, passing the port to
run on as an argument. This of course means your Threepenny app needs to take
the port as an argument. We suggest also setting stdout to be line buffered, at
least while developing. Altogether it should look something like this:

\begin{minted}[]{haskell}
module Main where

import System.Environment (getArgs)
import System.IO
import YourApp            (start)

main :: IO ()
main = do
    hSetBuffering stdout LineBuffering
    [port, otherArg1, otherArg2] <- getArgs
    start (read port)
\end{minted}

Now copy the linked \texttt{electron.js} to your project root directory. You'll have to
edit the defined constants: \texttt{relBin}, which is the relative path from
\texttt{electron.js} to your Threepenny application binary; and \texttt{binArgs}, which
contains any additional arguments to pass to the binary. If you're not sure
about the relative path to your application binary, and you're using Stack, see
the \hyperref[sec:orgcc92acc]{next section}.

Now run your app with Electron: \texttt{./node\_modules/.bin/electron electron.js}

\subsubsection{Explicit binary location}
\label{sec:orgcc92acc}
\texttt{relBin} is the relative path from \texttt{electron.js} to your Threepenny
application binary. This might change depending on which tool or
platform you are building with and thus can be a pain to set manually.
If you are using Stack you can easily build your application binary to
an explicit location, possibly a \texttt{build} directory:

\texttt{stack install -{}-local-bin-path build}

Now you can simply set \texttt{relBin} to \texttt{./build/your-app-exe}.

\subsection{Packaging with electron-packager}
\label{sec:orgfacb752}

This section assumes the app is already setup to run with Electron based on the
\hyperref[sec:org562eb1f]{above} instructions.

First install electron-packager: \texttt{npm install electron-packager}

Optionally edit the "name" field in \texttt{package.json} to set the name of the
packaged app. Then to package the app for the current platform, simply:

\begin{verbatim}
./node_modules/.bin/electron-packager .
\end{verbatim}

This is the most basic way to package the app, it will copy the current
directory to the packaged app. However you'll likely want to avoid copying some
source files, which can be achieved with the \texttt{-{}-ignore} flag. You might end up
using:

\begin{verbatim}
./node_modules/.bin/electron-packager . --ignore=app --ignore=src
\end{verbatim}

If you are using Stack and building your application binary to an explicit
location, as explained \hyperref[sec:orgcc92acc]{above} then you might want to also ignore \texttt{.stack-work/}.
An icon can be set by passing the icon path to \texttt{-{}-icon}, note that the icon
format \href{https://github.com/electron-userland/electron-packager/blob/master/docs/api.md\#icon}{depends on the platform}. For more options use the \texttt{-{}-help} flag.
\end{document}
